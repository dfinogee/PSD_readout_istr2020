


\documentclass[a4paper,11pt]{article}
 \pdfoutput=1 % if your are submitting a pdflatex (i.e. if you have
             % images in pdf, png or jpg format)
\usepackage{jinstpub} 


% for details on the use of the package, please
                     % see the JINST-author-manual



\title{The Readout system of the CBM Projectile Spectator Detector at FAIR}


\author[a,c,1]{D. Finogeev,\note{Corresponding author.}}
\author[a,b]{F. Guber,}
\author[a]{N. Karpushkin,}
\author[a,b]{A. Makhnev,}
\author[a,c]{S. Morozov,}
\author[a]{D. Serebryakov,}

\affiliation[a]{Institute for Nuclear Research RAS, Moscow, Russia,}
\affiliation[b]{Moscow Institute of Physics and Technology, Dolgoprudny, Moscow Region, Russia}
\affiliation[c]{National Research Nuclear University MEPhI, Moscow, Russia}
\affiliation[d]{ Joint Institute for Nuclear Research, Dubna, Russia}


% e-mail addresses: only for the corresponding author
\emailAdd{dmitry.finogeev@cern.ch}




\abstract{The Projectile Spectator Detector (PSD),  a sampling lead/scintillator forward hadron calorimeter with transverse and longitudinal segmentation and with MPPCs photodetectors, will be used at the Compressed Baryonic Matter (CBM) experiment at FAIR to measure the centrality and orientation of the reaction plane in nucleus-nucleus collisions. The CBM experiment will use a free-streaming data acquisition system (DAQ), which requires a coordinated time stamping of data in all sub-systems. The preparation for the CBM experiment starts from the mCBM setup at SIS18 accelerator in GSI, Darmstadt, Germany. The mCBM project is the first implementation of triggerless data readout from the prototypes of CBM detectors, which allows testing of electronics and detectors design in a close to real experimental conditions. A single PSD module (mPSD) has been integrated into the mCBM experiment at the SIS18 facility of GSI/FAIR joining the FAIR Phase-0 program. Details of the mPSD readout electronics and the first results of the data processing and transmission within the common, synchronized mCBM data transport taken during the data campaign in Q4/2019 -  Q1/2020 are shown. }



\keywords{Hadron calorimeters, trigger-less readout, GBT readout}

\collaboration[c]{on behalf of CBM collaboration}



\begin{document}
\maketitle
\flushbottom

\section{The CBM experiment at FAIR}
\label{sec:intro}
The future Compressed Baryonic Matter (CBM) experiment at the Facility for Antiproton and Ion Research (FAIR) is aimed to explore the Quantum Chromodynamics (QCD) phase diagram in the region of high baryon densities \cite{1}. The CBM will operate in the beam energy range of 2 - 11 AGeV and beam interaction rates up to 10 MHz. The CBM experiment will use a free-streaming data acquisition system (DAQ), which requires a coordinated time stamping of data in all sub-systems. This coordination includes a time-stamping against related clocks (common base), a synchronization procedure (deterministic time offsets) and the delivery of data containers of identical size during the data taking.

%At present, the mini CBM setup (mCBM) assembled of all CBM subdetectors prototypes is installed at SIS18 \cite{2}. The primary aim of mCBM is to study, commission and test the complex interplay of the different detector systems with the free-streaming data acquisition and the fast online event reconstruction and selection. In particular, it will allow to test the detectors and electronics components developed for the CBM experiment as well as the corresponding online/offline software packages under realistic experiment conditions up to top CBM interaction rates of 10 MHz. The mCBM is used as the first implementation of a combined readout with CBM subsystems of different front-end electronics types.

\section{The mCBM@SIS18 experiment}

The mCBM@SIS18 is a full-system test for CBM at the GSI/FAIR named mini-CBM (mCBM). The test setup include detector modules from all CBM detector subsystems (MVD, STS, RICH, MUCH, TRD, TOF, ECAL, PSD), using (pre-)series production specimen, positioned downstream of a nuclear target at an angle of 25◦ with respect to the beam axis. The concept of mCBM is sketched in figure~\ref{fig:1}

\begin{figure}[htbp]
\centering 
\includegraphics[width=.8\textwidth]{mCBM_sketch.png}
\caption{\label{fig:1} Concept sketch of the proposed mCBM test-setup}
\end{figure}

The mCBM setup will allow to test important aspects of CBM development \cite{2}: the operation of the detector prototypes in a high-rate nucleus-nucleus collision environment; the free-streaming data acquisition system including the data transport to a high-performance computer farm located in the Green IT Cube; the online track and event reconstruction as well as event selection algorithms; the offline data analysis; and the detector control system.
To make possible CBM@FAIR data collection with nucleus-nucleus collision rate up to 10MHz leading to data rate up to 1TB per second, will be used ultra-fast and radiation tolerant GBTx ASIC data aggregation unit developed at CERN. Future down-stream, the data streams are handled by Data Processing Boards (DPB) containing powerful FPGAs and are forwarded via FLES Input Selector (FLES) which performs on-line event selection. In 2020-2021 DPB and FLIB will be replaced by a prototype of the Common Readout Interface (CRI) as it foreseen in CBM experiment.



\section{The Projectile spectator detector at the CBM}
The forward hadron compensating lead/scintillator calorimeter - the Projectile Spectator Detector (PSD), with transverse and longitudinal segmentation and with the micropixel photodetectors light readout is one of the CBM subdetector, which
will be used in the CBM experiment to measure the centrality and the reaction plane orientation in heavy-ion collisions \cite{3}. The PSD will be assembled from already constructed 46 individual modules with the beam hole in the center, figure~\ref{fig:2} left. Each module consists of 60 lead/scintillator samples with 4 mm thickness of scintillator plate and 16 mm thickness of lead plate. Total length of module is about 5.6 nuclear interaction lengths. The transverse size of the module is 200x200 mm2
and the weight about 500 kg. Light from each scintillator plate is collected by the WLS-fiber glued in the circle grove in scintillator plate and stretched in 2 mm air gap at one the lateral side of the module. Light readout from each scintillator plate is provided by WLS-fibers embedded in grooves in the scintillator plates. Six consecutive scintillator tiles form a section and are collected together on separate optical connector at the end of the module. One Hamamatsu MPPC S12572-010P with active area 3×3mm2 is used as photodetector for each section of module. In addition, one optical fiber for the monitoring system is glued into the same optical connector in addition to these 6 WLS-fibers. Other ends of 10 such fibers are collected together on separate optical connector to be illuminated by the light emitting diode (LED). 


\begin{figure}[htbp]
\centering % \begin{center}/\end{center} takes some additional vertical space
\includegraphics[width=.2\textwidth]{PSD_CBM_sketch.png}
\qquad
\includegraphics[width=.2\textwidth]{PSD_module_sketch.png}
\qquad
\includegraphics[width=.2\textwidth]{PSD_module_photo.png}
\caption{\label{fig:2} Schematic view of the CBM PSD, left. Schematic view of the PSD module structure, center. Photo of assembled PSD module, right}
\end{figure}

Each photodetector is provided with a common reference voltage of about 70 volts, as well as individual automatic correction of the reference voltage depending on the external temperature. The temperature sensor, as well as the LED for the possibility of photoelectronic calibration of photodiodes, are placed on the FEE panel. The monitoring and data collection from all FEEs are carried out by means of the slow control system.
The monitoring system provides the MPPCs gain control during the data taking. Thus, each PSD module has ten longitudinal sections, which ensures the uniformity of light collection along the modules. Schematic view and photo of a PSD module without top cap are shown in figure~\ref{fig:2}, center and right.




\section{PSD readout concept}

Signal readout is accomplished with an ADC board represented in figure~\ref{fig:3} (left) and an Addon board represented in figure~\ref{fig:3} (right). Signals from MPPCs are collected by the ADC Addon board, which provides a single-ended interface based on single-ended to differential converters AD8138 ICs. Design allows for an adjustable input and output offset, utilizing the whole dynamic range of the converter. Besides providing a signal interface, Addon board includes an MPPC offset adjustment system, that consists out of 16 4-channel digital to analog converters (DACs) with unity-gain buffers. DACs provide a DC offset on the signal line, regulating voltage over the MPPC's terminals. Control of the DACs is accomplished through a STM32F103 microcontroller placed on the Addon board, which receives commands and adjustment data from the ADC board.

\begin{figure}[htbp]
	\centering
	\includegraphics[width=.3\textwidth]{ADC_board.png}
	\quad
\includegraphics[width=.4\textwidth]{ADC_addon.png}
	\caption{\label{fig:3} 64-channel ADC board with 2 FPGA Kintex 7 (left); Addon board design with the ADC board mounted}
\end{figure}

Signal is then digitized by the ADC board designed for the ECAL detector of PANDA experiment \cite{4} (see figure~\ref{fig:3}, left). The 64-channel board based on ADC LTM9011 with digitization rate up to 125Msps and 14-bit digitization resolution. The ADC board is mounted directly onto the Addon board (see figure~\ref{fig:3}, right), and the whole assembly is designed to fit standard 6U Eurocard cradles.
Digitized samples of 64 analog signals are sent to 2 FPGAs Kintex 7 using 128 LVDS links. Each FPGA process data from 32 channels. Will be use signal processing algorithm based on the Prony-LS fitting procedure, which allows working with signals near the noise level, and also used for the pile-up recognition. Such algorithm has already been developed and will be implemented in FPGA firmware in future.
To meet requirement of experiment DAQ, GBT FPGA transceiver was integrated into firmware. GBT allow to transmit data, slow control and clock. Expected data rate is 1MHz per channel that constrains 100bit/hit for GBT data rate 3.2 Gbit/s. 
LMK0460 jitter cleaner used for production ADC and MGTREF clocks. After reset signal, 100MHz clock from on-board generator TD-100 used as master clock. After GBT RX synchronization established, source switched to external clock sourced from GBT RX. The clock switching scheme allow to maintain common DAQ clock domain. Measuring hit time in trigger-less readout occurs relative to time counting named timeslice synchronous for all experiment readout.

\section{mCBM beam tests at 2019 - 2020}
At the end of 2019 and beginning of 2020 beam tests at mCBM@SIS18 was carried out with Au and Ur ions at 1.01 - 1.22 AGeV energy range. The main goal is intermediate tests of detectors subsystems, DAQ and online data aggregation procedure. mPSD prototype as part of mCBM experiment allow to approve and test the workability of the PSD readout concept. Prototype include crucial parts of the readout such as FEE addon and ADC FPGA readout board.
/// description of FEE ///
Photo of the assembled FEE setup is represented on figure~\ref{fig:4}.

\begin{figure}[htbp]
\centering % \begin{center}/\end{center} takes some additional vertical space
\includegraphics[width=.5\textwidth]{mPSD_FEE_photo.jpeg}
\qquad
\includegraphics[width=.3\textwidth]{mPSD_module_photo.jpeg}
\caption{\label{fig:4} Photo of readout ADC board assembled with FEE addon (left); photo of mPSD modulse connected to FEE}
\end{figure}

ADC FPGA board was connected to mCBM DAQ via GBT protocol used for data transport, clock distribution and configuration purposes. ADC was used in 80 Msps digitization mode, the setup will be upgraded up to 120 Msps in 2020. Each channel triggered independently by amplitude threshold crossing and provide charge in fixed gate and waveform of measured signal. ADC takes data in trigger-less readout according to CBM DAQ requirements. Also prototypes of all crucial software parts such data unpacker, event building and data monitor was developed and tested.

\subsection{PSD data monitoring}
Information of the mPSD detector is transmitted to the mCBM common data reading system using the DPB board with modified firmware. Currently, mPSD data packages are sets of 64-bit messages containing information about the time stamp of the signal, total number and indices of triggered channels, charges, zero levels and times of arrival of signals, as well as the information about the signal waveforms.
The completeness and integrity of the received data packets is monitored by the number of read GBT words. In addition, control of all transmitted information is also carried out online through a software module for data monitoring (see figure~\ref{fig:5}). The top two graphs serve here to track the indices of triggered channels. The indices correspond to the PSD sections from 0 to 8. An external source was connected to channel 9, which serves to check the synchronization between the mCBM detectors. The lower left graph shows the distribution of energy deposition in PSD sections. The major part of the energy is deposited in the two front sections of the calorimeter. The bottom right graph shows the evolution of the length of the microslices. This information is used to verify data integrity.

\begin{figure}[htbp]
\centering % \begin{center}/\end{center} takes some additional vertical space
\includegraphics[width=\textwidth]{run582.png}
\caption{\label{fig:5} Data monitoring software module}
\end{figure}


\subsection{mPSD data analysis}
It is one of the prime aims of mCBM to test and validate the data processing concept and the reconstruction software which are being developed for the full CBM experiment. mCBM thus will be a demonstrator for the computing concept of CBM, including the reconstruction of events and selection of data in real-time and the full offline data analysis. It is thus planned to use already existing software components as far as possible for both online and offline computing in mCBM.
The challenge of the trigger-less readout is time synchronization and event building procedure. It is necessary to construct events from signals based on their time stamps while data processing.
During the data acquisition, information from all the detectors of the mCBM experiment is recorded in a common binary file. To check the synchronise of data from the detectors, the time correlation graphs are constructed (see figure~\ref{fig:6}). An explicit peak in the distribution of the difference in the response time of the detectors T0 and PSD (time offset), located at about 200 ns, indicates the correlation of the data and serves to select beam events.
In the same way, the timestamps of signals from all subsystems are used to attribute these signals to a specific event. To ensure the possibility of this approach, time synchronization between the subsystems is used taking into account the time offsets that are inevitably introduced during the signal acquisition process. In the current approach, we assume the signals to form a single event, if their time stamps are located in a single time window of 200ns. In addition to introducing a fixed time window for event selection, conditions on the minimum number of signals from subsystems that are located in the considered time window can be applied. Such a straightforward approach is justified at the debugging stage and will be improved in the future.

\begin{figure}[htbp]
\centering % \begin{center}/\end{center} takes some additional vertical space
\includegraphics[width=\textwidth]{run582T0Psd.png}

\caption{\label{fig:6} T0-Psd time correlation}
\end{figure}

\section{Conclusions}
The prototype of PSD@CBM reaout system was implemented and tested in real beam and DAQ experimental condition. Prototype include all conceptual hardware, firmware and software parts which will be used for final PSD@CBM setup developement. Psychics data was taken inbeginning of 2020 at the beam test mCBM@SIS18.
The distribution of mPSD energy deposition in events constructed is shown in figure~\ref{fig:7} (left). The presence of two peaks here is explained by the discrete number of triggered sections in the event. The right side of figure~\ref{fig:7} shows the energy profile in the mPSD sections.

\begin{figure}[htbp]
\centering % \begin{center}/\end{center} takes some additional vertical space
\includegraphics[width=.45\textwidth]{PsdEdepInEvent_calibrd.png}
\qquad
\includegraphics[width=.45\textwidth]{PsdEprofileInEvent_calibrd.png}
\caption{\label{fig:7} mPSD energy deposition (left); mPSD energy profile (right)}
\end{figure}

ADC data processing on 80 Msps rate and GBT protocol functionality with clock switching procedure was tested. 
// conclusion for FEE //
Preliminary results of the analysis of data collected on the test beams of the mCBM experiment in the trigger-less mode were presented. The approach to tracking the integrity and quality of the data being collected is described, as well as the principle of the formation of events by the signal's time stamps is shown.
Workability of the readout concept was proved and development of full PSD readout system was started.

\acknowledgments
This work was supported by the Russian Foundation of Basic Research (RFBR) Grant No. 1 ??????????



\begin{thebibliography}{99}


\bibitem{1}
CBM Collaboration, T. Ablyazimov et al., \emph{Challenges in QCD matter physics -The scientific programme of the Compressed Baryonic Matter experiment at FAIR} Eur.Phys.J. A53 (2017) no.3, 60 


\bibitem{2}
C.Sturm et al., mCBM@SIS18, \emph{A CBM full system test-setup for high-rate nucleus-nucleus collisions at GSI / FAIR} http://www.fair-center.eu/fileadmin/fair/experiments/CBM/documents/mcbm-proposal2GPAC-WebVersion0619-SVN7729.pdf

\bibitem{3}
Guber F, et al., \emph{Technical Design Report for the CBM Projectile Spectator} https://repository.gsi.de/search?p=id:%22GSI-2015-02020%22

\bibitem{4}
Serneguet Sorli, Á. (2015). \emph{A multichannel digitizer for the PANDA experiment.} http://hdl.handle.net/10251/56722.



\end{thebibliography}
\end{document}

